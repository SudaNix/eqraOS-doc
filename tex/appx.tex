\documentclass[document.tex]{subfiles} 
\begin{document}

\appendix
\chapter{ترجمة وتشغيل البرامج}
\label{apx:compile_link}

لتطوير نظام التشغيل يجب استخدام مجموعة من الادوات واللغات التي تساعد وتيسير عملية التطوير وفي هذا الفصل سيتم عرض هذه الأدوات وكيفية استخدامها.\\

اعداد مترجم فيجوال سي++ لبرمجة النواة.
\section{نظام ويندوز}
\section{نظام لينوكس}

\chapter{المراجع}


\begin{english}
\fontspec[Scale=1.2,Mapping=englishdigits]{Calibri}
\begin{thebibliography}{10}
 
\bibitem{Stallings5}
  William Stallings,
  \emph{Operating System: Internals and Design Principles}.
  Prentice Hall,
  5th Edition,
  2004.
 
\bibitem{Tanenbauma}
  Andrew S. Tanenbaum ,Albert S Woodhull,
  \emph{Operating Systems Design and Implementation}.
  Prentice Hall,
  3rd Edition,
   2006.

\bibitem{Tischer}
  Michael Tischer, Bruno Jennrich,
  \emph{PC Intern: The Encyclopedia of System Programming}.
  Abacus Software,
  6th Edition,
  1996.

\bibitem{Messmer}
  Hans-Peter Messmer,
  \emph{The Indispensable PC Hardware Book}.
  Addison-Wesley Professional,
  4th Edition,
  2001.

\bibitem{Tanenbaumb}
  Andrew S. Tanenbaum,
  \emph{Structured Computer Organization}.
  Prentice Hall,
  4th Edition,
  1998.

\bibitem{Tanenbaumb}
  Ytha Yu,Charles Marut,
  \emph{Asssembly Language Programming and Organization IBM PC}.
  McGraw-Hill/Irwin,
  1st Edition,
  1992.

\bibitem{intel}
  Intel® Manuals,
  \emph{Intel® 64 and IA-32 Architectures Software Developer's Manuals}.
  \url{http://www.intel.com/products/processor/manuals/}


\bibitem{OSDev}
   \emph{OSDev}:
   \url{http://wiki.osdev.org}

\bibitem{brokenthorn}
  \emph{brokenthorn}:
  \url{http://brokenthorn.com}

\bibitem{sudancs}
  \emph{Computer Sciense Student's Community in Sudan}:
  \url{http://sudancs.com}

\end{thebibliography}
\end{english}

\chapter{شفرة نظام إقرأ}
كود النظام

\chapter{إتفاقية ترخيص المستندات الحرة \en{GNU FDL}}
\begin{english}
\fontspec[Scale=1.0,Mapping=englishdigits]{Calibri}
%\input{fdl-1.3}
\end{english}

\end{document}