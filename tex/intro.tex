\documentclass[document.tex]{subfiles} 
\begin{document}

\chapter*{مقدمة البحث}
تلعب أنظمة التشغيل دوراً مهما في شتى مجالات الحياة حيث أصبحت أحد أهم الركائز الأساسية لتشغيل وادارة أي جهاز أو عتاد يعتمد على الشرائح الالتكرونية المتكاملة ، فبدءا من جهاز الحاسب الشخصي و الجوالات و الأجهزة الكفية والمضمنة (\en{Embedded Device}) و أجهزة الألعاب والصرافات الآلية وحتى أجهزة الفضاء والدورات (\en{Orbiter}) كلها تعمل بأنظمة التشغيل. ونظراً لذلك فان مجال برمجة أنظمة التشغيل يعتبر من أهم المجالات في علوم الحاسب التي يجب أن تأخذ نصيبها من البحث العلمي والتطبيق البرمجي.

وعلى الرغم من أهمية هذا المجال الا انه يندر وجود بحوثا فيه ويعود ذلك لعدة أسباب: \textbf{الأول} هو  تنوع المجالات التي يجب دراستها قبل الخوض في برمجة نظام تشغيل حيث لا بد للطالب أو الباحث الإلمام بلغة السي والسي++ ولغة التجميع (\en{Assembly}) بالإضافة الى المعرفة التامة بمعمارية الحاسب من معالج وذاكرة ووحدات إدخال وإخراج. أما السبب \textbf{الثاني} فهو عدم توفر مراجع وكتبا باللغة العربية تشرح الأسس العلمية لبرمجة أنظمة التشغيل. والسبب \textbf{الثالث} هو توفر كمية كبيرة من أنظمة التشغيل في الوقت الحالي تجعل الطالب يعتقد بعدم الحوجة للبحث في هذا المجال وهذا مفهوم خاطئ حيث أن مبرمج نظام التشغيل ليس بالضرورة أن يبرمج نظاما من الصفر وانما يمكن أن يقوم بالتعديل والتطوير في أحد الأنظمة المفتوحة المصدر كذلك ربما يعمل في برمجة برامج النظام التي تتطلب الماما تاما بفاهيم أنظمة التشغيل مثل برمجة برامج اصلاح القاطاعات التالفة (\en{Bad Sectors}) واسترجاع الملفات المفقودة وغيرها من برامج النظام.

وقد وضع الكاتب نصب عينيه في هذا البحث التطرق للأمور البرمجية بتفاصيلها والتركيز على كيفية كتابة الشفرة لكل جزئية في نظام التشغيل . و لم يتم ذكر كل الجوانب النظرية في الموضوع وهذا بسبب أن الأمور النظرية في الغالب تأخذ بالطالب بعيدا وتحجب رؤيته عن حقيقة عمل نظام التشغيل . وقد تم برمجة النظام من الصفر دون الإعتماد على أي مكونات أو شفرات جاهزة (مفتوحة المصدر) ولا يمكن اعتبار هذا إعادة اختراع للعجلة ! بل هو أساس يمكن الاعتماد عليه وتعليم الطلاب عليه وهكذا يتطور المشروع ويتقدم الى الأمام وفي نفس الوقت تزداد خبرة الطالب العملية في المجال.


 و يجدر بنا ايضاح أن هدف النظام (نظام إقرأ) هو للاستخدامات الأكاديمية والتعليمية وليس للمستخدم الأخير ، حيث أن الهدف هو تعليم الطالب على هذه الأداة واعداده للعمل على أنظمة ضخمة مثل جنو/لينوكس.\\\\

 \small{
أحمد عصام عبد الرحيم.\\
29/6/2010
}
\end{document}