\documentclass[document.tex]{subfiles} 
\begin{document}

\chapter*{ملخص البحث}
 الهدف الرئيسي لهذه الدراسة هو توفير بحثاً متكاملاً للطالب والباحث في مجال برمجة أنظمة التشغيل وذلك لإفتقار المكتبة العربية لهذه النوعية من البحوث ، و تناولت هذه الدراسة  العديد من أساسيات ومفاهيم نظم التشغيل (إقلاع النظام ، الإنتقال الى النمط المحمي ، إدارة الذاكرة ، أنظمة الملفات وتعريفات العتاد) وكيفية برمجتها من الصفر دون الإعتماد على أي مكونات أو مكتبات خارجية. ويحوي هذا البحث شفرة نظام تشغيل ( تم تسميته بنظام \textbf{إقرأ}) والذي تم برمجته ليكون عوناً ودليلاً للطالب أثناء دراسته في هذا المجال.
 و رؤية الباحث في هذه الدراسة هي أن تستخدم كمنهج لتدريس الطلاب في مادة نظم التشغيل وأن تُدرَّس الشفرة المصدرية للنظام . ولا يُقتصر على ذلك بل يَستمر التطوير في النظام ليكون أداة تعليمية (مفتوحة المصدر) للطلاب والباحثين.


\end{document}