\documentclass[document.tex]{subfiles} 
\begin{document}

\chapter*{ملخص البحث}
هدف هذا البحث هو برمجة نظام تشغيل (نظام إقرأ) للإستخدامات التعليمية والأكاديمية بحيث يُمكِّن الطالب من تطبيق ما تعلمه من نظريات خلال فترة الدراسة على نظام تشغيل حقيقي مما يكسبه خبرة ويؤهله للمشاركة في برمجة أنظمة تشغيل ضخمة مثل جنو/لينوكس. كذلك يهدف الى توفير بحثا باللغة العربية يشرح الأسس العلمية لكيفية برمجة نظام تشغيل من الصفر دون الإعتماد على أي مكونات خارجية في الوقت الذي تندر توفر مثل هذه البحوث المفيدة للطالب وخاصة في هذا المجال الذي يعتبر من أهم المجالات في علوم الحاسوب.\\

وتناولت هذه الدراسة العديد من أساسيات ومفاهيم نظم التشغيل بدءا من مرحلة إقلاع الحاسب والإنتقال الى النمط المحمي وانتهاءا ببرمجة مديرا للذاكرة وتعريفات لبعض العتاد.\\
 
 و رؤية الباحث في هذه الدراسة هي أن تستخدم كمنهج لتدريس الطلاب في مادة نظم التشغيل وأن تُدرَّس الشفرة المصدرية للنظام . ولا يُقتصر على ذلك بل يَستمر التطوير في النظام ليكون أداة تعليمية (مفتوحة المصدر) للطلاب والباحثين.

\end{document}