\documentclass[document.tex]{subfiles} 
\begin{document}

\chapter*{الخاتمة}
وهكذا نصل الى نهاية المطاف لهذا البحث المتواضع والذي نرجو أن يكون إضافة ولو يسيرة للمكتبة العربية وللقارئ العربي في مجال برمجة أنظمة التشغيل. ونسأل الله تعالى أن ينفعنا بما علمنا وأن ينفع بنا الإسلام والمسلمين وأن يعيننا على تعلم العلم وتبليغه انه ولي ذلك والقادر عليه. وأخيرا ما كان من خطأ فمن نفسي والشيطان وما كان من صواب فمن الله عز وجل.\\

وصلى اللهم وسلم على نبينا محمد وعلى آله وصحبه أجمعين . وآخر دعوانا أن \textbf{الحمد لله رب العالمين}.

\section*{النتائج}


\section*{التوصيات}
حتى يصل البحث لمرحلة مناسبة فانه يجب الحديث عن عددا من الموضيع التي تم تجاهلها لسبب أو لآخر ، فبداية من الفصل الأول والذي يعتبر مقدمة عامة الى أنظمة التشغيل حيث يمكن أن تضاف له العديد من المعلومات عن الأنظمة الحالية وكذلك مزيدا من التوضيحات عن ماهية أنظمة التشغيل. أما الفصل الثاني فلم يتحدث سوى عن القشور في معمارية الحاسب حيث أنه يتطلب كتابا متكاملا لشرح كل جزئية ولكن تم الإيجاز والحديث عن المواضيع التي يكثر تناولها عند برمجة أنظمة التشغيل.  وقبل الإنتقال الى الفصل الثالث يمكن كتابة فصل جديد عن لغة التجميع بحيث يشرح أساسيات اللغة مع مترجم \en{NASM} لانه المترجم المستخدم في عملية تجميع البرامج حيث أنه يحوي أوامر خاصة به يجب الوقوف عندها وتوضيحها بشكل جيد. أما الفصل الثالث والرابع فتم الحديث فيها عن محمل النظام الذي تم برمجته لتحميل نظام إقرأ ويمكن هنا الحديث عن أي من المحملات المفتوحة المصدر (مثل محمل \en{GRUB} والذي يعتبر من أشهر المحملات وأكثرهم استخداما) وكيفية تحميل نواة النظام مباشرة من خلالهم. والفصل الخامس تحدث عن النواة وطرق تصميمها بشكل مختصر ولم يتناول الاختلاف بينهم وطرق برمجة كل منهم على حدة ويعتبر هذا الفصل مجالا كاملا للبحث فيه .  أما الفصل السادس فهو مكتمل تقريبا ويمكن اضافة المزيد من العلومات ، بينما الفصل السابع لم يتحدث عن خوارزميات ادراة الذاكرة وتم تطبيق خوارزمية \en{First Fit} للبحث عن أماكن شاغرة في الذاكرة ويمكن تطبيق أي من الخوارزميات الشهيرة ، كذلك يمكن تطبيق القائمة المتصلة لمتابعة الأماكن الشاغرة في الذاكرة بدلا من استخدام \en{Bit Map}. أما الفصل الثامن فهو عن مشغلات الأجهزة لكن نظرا لضيق الوقت فقد تم الحديث عن متحكم لوحة المفاتيح وكيفية قراءة الأحرف المدخلة منه ، ويمكن في هذا الفصل اضافة العديد من مشغلات الأجهزة مثل مشغل القرص المرن والقرص الصلب ومشغل لكرت الشبكة وكرت الصوت والعديد من المشغلات الضرورية. أخيرا يمكن إضافة فصل للحديث عن أنظمة الملفات وكيفية برمجتهم مثل \en{FAT12,FAT16,FAT32,EXT3,EXT4,...etc} وفصل لبرمجة المجدول لدعم تعدد المهام.\\\\

ومن ناحية نظام \textbf{إقرأ} فانه بحاجة الى العديد من الإضافات والتحسينات:
\begin{itemize}
\item دعم واجهة \en{POSIX}.
\item دعم لأنظمة الملفات \en{FAT12,FAT16,FAT32,EXT3,EXT4,...etc}.
\item دعم لتعدد المهام.
\item نقل (\en{port}) مترجم \en{gcc,g++} وواجهة \en{x11} الى نظام \textbf{إقرأ}.

\end{itemize} 
\end{document}